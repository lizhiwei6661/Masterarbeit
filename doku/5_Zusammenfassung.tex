\chapter{Zusammenfassung und Ausblick}

Diese Masterarbeit behandelte zwei zentrale Aspekte der spektralen Reflexionsmessung: die Neuentwicklung der Aleksameter-Software sowie die wissenschaftliche Untersuchung des Einflusses von Wellenlängenbegrenzungen auf die Farbgenauigkeit. Die ursprüngliche MATLAB-basierte Software wurde vollständig neu entwickelt und zu Python migriert, um kritische Schwachstellen der ursprünglichen Implementierung zu adressieren. Der Übergang zu Python mit wissenschaftlichen Bibliotheken und Qt-basierter Benutzeroberfläche eliminierte Lizenzkosten und Plattformabhängigkeiten. Die neue Architektur folgt dem MVC-Designmuster und gewährleistet vollständige numerische Kompatibilität. Das adaptive Layout mit integrierten Spektraldarstellungen, Chromatizitätsdiagramm und farbkodierten Ergebnistabellen verbessert die wissenschaftliche Datenvisualisierung erheblich. Die Software unterstützt verschiedene Betriebsmodi, Batch-Verarbeitung, multiple Exportformate und publikationsreife Grafiken.

Die systematische Analyse repräsentativer Oberflächenproben untersuchte die Auswirkungen der Verkürzung des spektralen Messbereichs auf die Farbgenauigkeit. Die ermittelten Farbdifferenzen liegen deutlich unter der menschlichen Wahrnehmungsschwelle. Die Farbfehler entstehen hauptsächlich durch den Verlust spektraler Information in den Randbereichen, wo das menschliche Auge wenig empfindlich ist, aber dennoch zur Gesamtfarbwahrnehmung beiträgt. Die entwickelte lineare Extrapolationsmethode reduziert die Farbfehler erheblich und zeigt für alle untersuchten Proben deutliche Verbesserungen.

Für Standardanwendungen ist ein verkürzter spektraler Messbereich für die meisten praktischen Anwendungen ausreichend, da die resultierenden Farbfehler vernachlässigbar klein sind. Für Präzisionsmessungen sollte der vollständige spektrale Bereich verwendet werden, insbesondere bei neutralen Proben, die tendenziell stärker von Wellenlängenbegrenzungen betroffen sind. Für die Geräteentwicklung empfiehlt sich die Integration der linearen Extrapolation als Standardfunktion.

Diese Arbeit leistet sowohl praktische als auch theoretische Beiträge zur Spektralmesstechnik. Softwaretechnisch stellt sie eine moderne, plattformunabhängige Lösung für spektrale Farbmessungen bereit. Wissenschaftlich bietet sie eine quantitative Bewertung der Auswirkungen von Wellenlängenbegrenzungen und die Entwicklung einer effektiven Kompensationsmethode. Methodisch validiert sie eine Migrationsstrategie von proprietärer zu Open-Source-Software bei Beibehaltung der numerischen Genauigkeit. Die Arbeit eröffnet Ansatzpunkte für weiterführende Forschung, einschließlich erweiterter Extrapolationsmethoden, materialspezifischer Analysen und adaptiver Messstrategien. Die Ergebnisse tragen zur besseren Einschätzung der praktischen Anwendbarkeit spektraler Messgeräte mit eingeschränktem Wellenlängenbereich bei und bieten konkrete Lösungsansätze zur Kompensation entstehender Messfehler. 